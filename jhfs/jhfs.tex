\documentclass{article}
\usepackage[utf8]{inputenc}

\title{IF684 - Sistemas Inteligentes}
\author{José Henrique Fernandes Silva}

\usepackage{natbib}
\usepackage{graphicx}
\usepackage{float}
\usepackage{indentfirst}
\usepackage[portuguese]{babel}

\begin{document}

\maketitle

\section{Introdução}
    A disciplina de Sistemas Inteligentes (IF684), busca cobrir os principais assuntos, necessários, para a compreensão e introdução dos alunos a inteligência artificial (IA), sendo ministrada pelos professores Sergio Queiroz e Cleber Zanchettin.
    \citep{primeira, segunda, terceira}
    
    Se inserindo na área da Inteligência Artificial, e tendo como os principais tópicos cobertos por ela: Modelos de agentes inteligentes, que apresenta o conceito do que seria agentes inteligentes e, quais são os principais modelos, por exemplo, agentes reativos simples, agentes reativos baseado em modelo, agentes cognitivos, agentes baseados em objetivos e agentes baseados na utilidade; Busca (I, II, III e IV), que apresenta os conceitos de busca cega, busca heurística, funções heurísticas e busca com otimização; Programação de seleção de clones (CSP), que é um método de programação proposto para melhorar a eficácia da codificação de programas e dos mecanismos de busca; Sistemas baseados em conhecimento: que no geral são representados por programas de computadores que usam o conhecimento representado explicitamente para resolver um problema; Introdução a aprendizado de máquinas, que estuda a capacidade dos computadores de aprenderem sem serem explicitamente programados; Redes neurais, que são sistemas de computadores com nós interconectados que funcionam como neurônios humanos, usando algoritmos eles podem reconhecer padrões escondidos e correlações em dados brutos; Sistemas difusos, que usam da lógica difusa (fuzzy), para resolver problemas com conhecimentos incompletos, incertos ou imprecisos.
    \cite{primeira, quarta, quinta, sexta, setima, oitava, nona}
    
    \begin{figure}[H]
        \centering
        \includegraphics[scale=0.26]{sistemas_inteligentes.jpg}
        \caption{Sistemas Inteligentes}
        \label{sistemas_inteligentes}
    \end{figure}

\section{Relevância}
    Em decorrência da breve apresentação da disciplina, fica evidente a sua relevância para o curso de Ciências da Computação, já que é responsável por da introdução a conceitos de alta relevância para a área da Inteligencia Artificial, que é uma área em grande ascensão na computação, pois já é capaz de resolver problemas reais, com uma grande capacidade de ser uma grande aliada para que possamos entender e resolver questões, que hoje são consideras, como impossíveis. 

    

\section{Relação com outras disciplinas}
    A seguir seguir segue uma tabela mostrando a relação de Sistemas Inteligentes com outras disciplinas do perfil curricular de Ciência da Computação. A primeira coluna da tabela faz referência ao código das disciplinas que se relacionam com Sistemas Inteligentes, já a segunda coluna é uma breve explicação sobre a relação das duas disciplinas.
    \begin{center}
    \begin{tabular}{|c|p{10cm}|}
        \hline
        Código & Relações \\ \hline
        IF752 & Em visão computacional, é necessário o conhecimento de sistemas inteligentes para servir como a construção de sistemas que obtém informações de imagens. \\ \hline
        IF809 & Em Top. Avanc. Robot. Autom. Inteligente, é extremamente necessário dominar os conceitos de Sistemas Inteligentes, já que uma disciplina é uma especialização da outra. \\ \hline
        IF702 & A disciplina de Redes Neurais, é uma estudo mais profundo de um tópico presente em  Sistemas Inteligentes.\\ \hline
        
    \end{tabular}
    \end{center}

\bibliographystyle{ieeetr}
\bibliography{references}
\end{document}
